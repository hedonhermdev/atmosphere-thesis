%%%%%%%%%%%%%%%%%%%%%%%%%%%%%%%%%%%%%%%%%%%%%%%%%%%%%%%%%%%%%%%%%%%%%%%%%%%%%%%%
% Thesis / Project Report
% LaTeX Template
% Version 2.0 (08/04/16)
%
% Author:
% Siddhant Shrivastava
% https://github.com/sidcode/bits-pilani-thesis-template-latex
%
% This template is heavily based on the work of Darshit Shah, Steven Gunn and Sunil Patel
% Darshit Shah
% https://github.com/darnir/BPHC-LaTeX-Report-Class
% Steven Gunn
% http://users.ecs.soton.ac.uk/srg/softwaretools/document/templates/
% and
% Sunil Patel
% http://www.sunilpatel.co.uk/thesis-template/
%
% License:
% CC BY-NC-SA 4.0 (http://creativecommons.org/licenses/by-nc-sa/4.0/)
%
% Note:
% Make sure to edit document variables in the Thesis.cls file
%
%%%%%%%%%%%%%%%%%%%%%%%%%%%%%%%%%%%%%%%%%%%%%%%%%%%%%%%%%%%%%%%%%%%%%%%%%%%%%%%%

%-------------------------------------------------------------------------------
%	PACKAGES AND OTHER DOCUMENT CONFIGURATIONS
%-------------------------------------------------------------------------------

\documentclass[11pt, a4paper, oneside]{Thesis} % Paper size, default font size
                                               % and one-sided paper

\graphicspath{{Pictures/}} % Specifies the directory where pictures are stored

% \usepackage{lineno} 
% \linenumbers % TODO: Line numbers in draft mode. Disable before submitting

% \usepackage[sfdefault]{atkinson} % atkinson hyperlegible font

\usepackage[backend=bibtex]{biblatex} % bibliography in the end
\bibliography{bibliography}

\usepackage{nameref}
\usepackage{minted}

\title{\ttitle} % Defines the thesis title - don't touch this

\begin{document}

\frontmatter % Use roman numbering style (i, ii...) for the pre-content pages

\setstretch{1.3} % Line spacing of 1.3

% Define page headers using FancyHdr package and set up for one-sided printing
\fancyhead{} % Clears all page headers and footers
\rhead{\thepage} % Sets the right side header to show the page number
\lhead{} % Clears the left side page header

\pagestyle{fancy} % Finally, use the "fancy" page style to implement the
                  %FancyHdr headers

% Input all the variables used in the document. Please fill out the
% variables.tex file with all your details.
%-------------------------------------------------------------------------------
%	DOCUMENT VARIABLES
%
%	Fill in the lines below to set the various variables for the document
%-------------------------------------------------------------------------------

%-------------------------------------------------------------------------------
% Your thesis title - this is used in the title and abstract
% Command: \ttitle
\thesistitle{Fine-Grained Isolation of the TCP/IP Stack with Language Safety}
%-------------------------------------------------------------------------------
% The document type: Thesis / report, etc.
% Command: \doctype
\documenttype{Undergraduate Thesis}
%-------------------------------------------------------------------------------
% Your supervisor's name - this is used in the title page
% Command: \supname
\supervisor{Dr. Anton Burtsev}
%-------------------------------------------------------------------------------
% The supervisor's position - Used on Certificate
% Command: \suppos
\supervisorposition{Asst. Professor}
%-------------------------------------------------------------------------------
% Supervisor's institute
% Command: \supinst
\supervisorinstitute{University of Utah}
%-------------------------------------------------------------------------------
% Your Co-Supervisor's name
% Command: \cosupname
\cosupervisor{Dr. K Hari Babu}
%-------------------------------------------------------------------------------
% Co-Supervisor's Position - Used on Certificate
% Command: \cosuppos
\cosupervisorposition{Asst. Professor}
%-------------------------------------------------------------------------------
% Co-Supervisor's Institute
% Command: \cosupinst
\cosupervisorinstitute{BITS Pilani Pilani Campus}
%-------------------------------------------------------------------------------
% Your Examiner's name. Not currently used anywhere.
% Command: \examname
\examiner{}
%-------------------------------------------------------------------------------
% Name of your degree
% Command: \degreename
\degree{Bachelor of Engineering (Hons.)}
%-------------------------------------------------------------------------------
% The BITS Course Code for which this report is written
% COmmand: \ccode
\coursecode{BITS F421T}
%-------------------------------------------------------------------------------
% The name of the Course
% Command: \cname
\coursename{Thesis}
%-------------------------------------------------------------------------------
% Your name. Extend manually in case of multiple authors
% Command: \authornames
\authors{Tirth Jain}
%-------------------------------------------------------------------------------
% Your ID Number - used on the Title page and abstract
% Command: \idnum
\IDNumber{2019A7TS0120P}
%-------------------------------------------------------------------------------
% Your address
% Command: \addressnames
\addresses{}
%-------------------------------------------------------------------------------
% Your subject area
% Command: \subjectname
\subject{}
%-------------------------------------------------------------------------------
% Keywords for this report.
% Command: \keywordnames
\keywords{Microkernels, Networking, Fault Isolation, TCP/IP}
%-------------------------------------------------------------------------------
% University details
% Command: \univname
\university{\texorpdfstring{\href{http://www.bits-pilani.ac.in/} % URL
                {Birla Institute of Technology and Science Pilani}} % University name
                {Birla Institute of Technology and Science Pilani}}
%-------------------------------------------------------------------------------
% University details, in Capitals
% Command: \UNIVNAME
\UNIVERSITY{\texorpdfstring{\href{http://www.bits-pilani.ac.in/} % URL
                {BIRLA INSTITUTE OF TECHNOLOGY AND SCIENCE PILANI}} % name in capitals
                {BIRLA INSTITUTE OF TECHNOLOGY AND SCIENCE PILANI}}

%-------------------------------------------------------------------------------
% Campus Name
% Command: \campusname
\campus{Pilani Campus}

%-------------------------------------------------------------------------------
% Campus Name, in capitals
% Command: \CAMPUSNAME
\CAMPUS{PILANI CAMPUS}


%-------------------------------------------------------------------------------
% Department Details
% Command: \deptname
\department{\texorpdfstring{\href{http://www.bits-pilani.ac.in/pilani/computerscience/ComputerScience} % Your department's URL
                {Computer Science \& Information Systems}} % Your department's name
                {Computer Science}}
%-------------------------------------------------------------------------------
% Department details, in Capitals
% Command: \DEPTNAME
\DEPARTMENT{\texorpdfstring{\href{http://www.bits-pilani.ac.in/pilani/computerscience/ComputerScience} % Your department's URL
                {COMPUTER SCIENCE \& INFORMATION SYSTEMS}} % Your department's name in capitals
                {COMPUTER SCIENCE \& INFORMATION SYSTEMS}}
%-------------------------------------------------------------------------------
% Research Group Details
% Command: \groupname
\group{\texorpdfstring{\href{https://mars-research.github.io/}
                {Mars Research}} % Your research group's name
                {Mars Research}}
%-------------------------------------------------------------------------------
% Research Group Details, in Capitals
% Command: \GROUPNAME
\GROUP{\texorpdfstring{\href{https://mars-research.github.io/}
                {MARS RESEARCH}}
                {MARS RESEARCH}}
%-------------------------------------------------------------------------------
% Faculty details
% Command: \facname
\faculty{\texorpdfstring{\href{https://www.cs.utah.edu/~aburtsev/index.html}
                {Anton Burtsev}}
                {Anton Burtsev}}
%-------------------------------------------------------------------------------
% Faculty details, in Capitals
% Command: \FACNAME
\FACULTY{\texorpdfstring{\href{Faculty Web Site URL Here (include http://)}
                {ANTON BURTSEV}}
                {ANTON BURTSEV}}
%-------------------------------------------------------------------------------


%-------------------------------------------------------------------------------
%   NON-CONTENT PAGES
%-------------------------------------------------------------------------------
\maketitle
\Declaration
\Certificate

% \Quotation{Insert Random Quote here. Publish like a boss.}{Your Name}

\begin{abstract}
Even after decades of work to make monolithic kernels more secure, serious vulnerabilities in them are still reported every year. Because the entire monolithic kernel is in one address space, an attacker is just one vulnerability away from owning the entire machine. We argue that it is time to decompose monolithic kernels like Linux into smaller parts that run in isolated compartments and communicate using secure interfaces. We think this is timely due to growing hardware and software support of isolation.

In this Thesis, I discuss Atmosphere, our approach to microkernelization and fault isolation of an operating system kernel. Specifically, my work focuses on the network stack which is part of our larger effort to build a new operating system, Atmosphere. We argue that the network stack is a source for bugs and that isolation is the way forward to minimize the impact of these bugs. 
\end{abstract}

% \begin{acknowledgements}
% Thanks to Anton for calling me to Utah and giving me this opportunity. Thanks to Prof. Hari Babu for teaching me NetProg. Thanks to Xiangdong for dragging me to the gym everyday and all of his compilers. Thanks to Zhaofeng for all the Nix shells and gatekeeping all our code. Without all of them, this Thesis wouldn't have been possible.
% \end{acknowledgements}

%-------------------------------------------------------------------------------
%	LIST OF CONTENTS/FIGURES/TABLES PAGES
%-------------------------------------------------------------------------------

% The page style headers have been "empty" all this time, now use the "fancy"
% headers as defined before to bring them back
\pagestyle{fancy}

\lhead{\emph{Contents}} % Set the left side page header to "Contents"
\tableofcontents % Write out the Table of Contents

% Set the left side page header to "List of Figures"
\lhead{\emph{List of Figures}}
% \listoffigures % Write out the List of Figures

 % Set the left side page header to "List of Tables"
\lhead{\emph{List of Tables}}
% \listoftables % Write out the List of Tables

%%-------------------------------------------------------------------------------
%%	ABBREVIATIONS
%%-------------------------------------------------------------------------------

%\clearpage % Start a new page

% % Set the line spacing to 1.5, this makes the following tables easier to read
%\setstretch{1.5}

%\lhead{\emph{Abbreviations}} % Set the left side page header to "Abbreviations"
%\listofsymbols{ll} % Include a list of Abbreviations (a table of two columns)
%{
%\textbf{LAH} & \textbf{L}ist \textbf{A}bbreviations \textbf{H}ere \\
%%\textbf{Acronym} & \textbf{W}hat (it) \textbf{S}tands \textbf{F}or \\
%}

%-------------------------------------------------------------------------------
%	PHYSICAL CONSTANTS/OTHER DEFINITIONS
%-------------------------------------------------------------------------------

% \clearpage % Start a new page

% % Set the left side page header to "Physical Constants"
% \lhead{\emph{Physical Constants}}

%  % Include a list of Physical Constants (a four column table)
% \listofconstants{lrcl}
% {
% Speed of Light & $c$ & $=$ & $2.997\ 924\ 58\times10^{8}\ \mbox{ms}^{-\mbox{s}}$ (exact)\\
% % Constant Name & Symbol & = & Constant Value (with units) \\
% }

%%-------------------------------------------------------------------------------
%%	SYMBOLS
%%-------------------------------------------------------------------------------

%\clearpage % Start a new page

%\lhead{\emph{Glossary}} % Set the left side page header to "Symbols"

%\listofnomenclature % List the nomenclature. (We use the glossaries package)

%-------------------------------------------------------------------------------
%	DEDICATION
%-------------------------------------------------------------------------------

\setstretch{1.3} % Return the line spacing back to 1.3

\pagestyle{empty} % Page style needs to be empty for this page

% Dedication text
% \Dedicatory{Dedicated to Kenkin, the emotional support capybara.}

\addtocontents{toc}{\vspace{2em}} % Add a gap in the Contents, for aesthetics

%-------------------------------------------------------------------------------
%	THESIS CONTENT - CHAPTERS
%-------------------------------------------------------------------------------

\mainmatter % Begin numeric (1,2,3...) page numbering

\pagestyle{fancy} % Return the page headers back to the "fancy" style

% Include the chapters of the thesis as separate files from the Chapters folder
% Uncomment the lines as you write the chapters

% Chapter 1

\chapter{Introduction} % Main chapter title

\label{Chapter1} % For referencing the chapter elsewhere, use \ref{Chapter1} 

\lhead{Chapter 1. Introduction} % This is for the header on each page - perhaps a shortened title

Operating system reliability is a topic that has been studied for decades but nevertheless remains a major concern today. Even thought its been 30 years since Linux kernel was developed, critical vulnerabilities are being found in the kernel. On top of that, the monolithic design (i.e., everything in the kernel is running in a single address space) makes it such that the attacker is only one exploit away from taking control of the entire system. This is concerning because device drivers, often developed by third parties are a major source of vulnerabilities. Redundancy in hardware and software protection mechanisms can protect against transient faults but persistent faults because of undetected logic flaws still remain a major concern. In recent times, with the advent of cloud services, this becomes even more significant because clouds often have to run untrusted code that needs to be isolated from other parts of the operating system.

Logic errors and faults can propogate from within a component and propogate to other parts of the system. Microkernelization aims to solve this problem by isolating various parts of an operating system and minimizing the effect of failures in a single subsystem. In a microkernel, only the bare minimum that is required to boot an operating system is included in the kernel. All other components such as the filesystem, the device drivers, and the network stack are run in separate isolated processes that interact with each other using some form of IPC. Inter component fault propogation can thus be reduced by introducing safe IPC mechanisms.

Atmosphere is a microkernel based operating system. In this thesis, I discuss the design of Atmosphere mainly focusing on the the network stack and how its design can be used to model other services of the operating system. The network stack was built with two main principles in mind: maximizing fault isolation and minimizing its cost on performance. For the first principle, we take microkernelization to its extremes and build a per connection (or per socket for UDP) network stack. This means, every new connection has its own TCP/IP stack that can process and dispatch packets. This ensures that if one connection is corrupted, the rest of the connections on the system can keep operating. Although this is not completely possible since there is some inherent shared state on any host machine on a network. In later sections we discuss how we minimize this shared state. For the second principle, we implement a zero copy model and explain how using Rust's guarantees help us implement this with confidence. On failure, a component can be safely restarted and meanwhile, all IPC calls to it can return an error. 

The rest of this Thesis is organised as follows: 

\begin{itemize}
    \item{In \nameref{Chapter2} we take a look at the contemporary approaches to microkernelization. }
    \item{In \nameref{Chapter3} we take a detour to discuss modern software and hardware isolation mechanisms and why we choose Rust. }
    \item{\nameref{Chapter4} discusses the design principles of Atmosphere. }
    \item{\nameref{Chapter5} talks about the API design and the internals of the network stack we built. }
    \item{\nameref{Chapter6} evaluates the performance of the network stack as compared to the current standards. }
\end{itemize}

% Chapter Template

\chapter{Background and Related Work}

\label{Chapter2}

\lhead{Chapter 2. \emph{Backround and Related Work}} 

\tirth{some intro here? idk}

\section{Theseus}
Theseus\cite{theseus} is a microkernel based operating system written in Rust. Rust's safety guarantees enable Theseus to run all software written in Rust, including userspace applications, to run in a single address space and at a single privilege level. Thus, eliminating the need for virtual memory management and protection rings. Theseus introduces the idea of cells that are described as a software-defined unit of modularity that serves as the core building block of the OS. A cell can be modeled as a crate in Rust. On booting, only the microkernel, ie, the \lstinline{nano_core} is loaded which bootstraps the system. All other cells are dynamically loaded on demand. 
Theseus piggybacks on Rust's safety guarantees to enable reliable IPC between cells. For example, memory mappings are a 1-1 mapping to a physical frame. They can only be shared behind a read-only \lstinline{&MappedPages} reference eliminating the double-free and the use-after-free problem. 

The most important contribution by Theseus is the idea of eliminating (or minimizing) shared state between components. All OS services can be modelled as servers and applications requesting these services are modelled as clients. Theseus eliminates state-spill\cite{state-spill} across cells by eliminating all state from the servers. Everything that is needed to service a client's request is stored in the client itself and thus the servers can be completely stateless. This allows a server cell to be replaced by a new cell without any state loss. This, however, is not always possible. In case a state cannot be eliminated, as in the case of descriptor tables, the state is stored in a \lstinline{state_db}. The \lstinline{state_db} is a key-value database that stores states with a static lifetime that the server can request a (weak) reference to. In case a server needs to be restarted, its state can be recovered from the \lstinline{statedb}. The statelessness of cells also allows for live updates. A server can be replaced, ie, a patch can be applied to a component without having to restart the entire operating system. Applying a patch is as simple as swapping a cell with a new cell.

\tirth{Maybe add limitations and performance evaluations of Theseus here?}

\section{Singularity OS}
Singularity\cite{singularity} introduced "Software Isolated Processes" (SIPs) which use software verification instead of hardware protection mechanisms to isolate processes. SIPs cannot have shared memory. Instead, data can be passed between SIPs using an "exchange heap". Data on the exchange heap can be owned only by a single process but the ownership can be "transferred". Static verification ensure that programs do not try to access an object after it has been passed (ie a dangling pointer). Ownership can be transferred between SIPs using "Contract-Based Channels". Channels are described using statically defined interfaces in the Sing\# language. The communicating SIPs act as state machines with clearly defined states and the messages that can be passed on each state. Once a message has been passed, its data can no longer be used by the sending SIP. Ownership of data on the exchange heap is recorded so that blocks can be freed on process termination preventing memory leaks. This process isolation allows singularity to run the kernel and all SIPs in a single physical address space.

All programs running on Singularity must ship with a manifest. Manifest-based programs clearly define their resource requirements, desired capabilities and dependencies on other programs. The manifest can be used by the system to ensure that the program's requirements can be met and that the program satisfies all correct usage guarantees. The absence of shared memory and static verification of all communication makes the creation and termination of SIPs inexpensive.

\section{RedLeaf}
RedLeaf is a microkernel written in Rust that aims to use Rust's language safety features to ensure fault isolation. RedLeaf forbids the use of shared data between components. Similar to Singularity, data can only be passed between domains using the "Exchange Heap". A domain cannot hold a pointer to an object in a different domain. Redleaf introduces \lstinline{RRefs} for passing data between domains. RRefs are smart-pointers that track the ownership of the allocated object. Along with the pointer, RRefs store the owner of an object and it's current reference count. Types that can be safely shared on the exhange heap are called exchangeable types and only exchangeable types can be passed across domains. Exchangeable types enforce the invariant that objects on the shared heap cannot contain any references to private or shared heap but can contain RRefs pointing to other objects on the shared heap. Rust's ownership discipline enforces that once an object is passed, there are no aliases left in the domain referring to the object. 

Cross-domain calls in RedLeaf are proxied with the help of invocation proxies. The invocation proxy first checks the liveliness of the domain being called. Then, the ownership of all RRefs is moved to the callee domain. RedLeaf also introduces an IDL to validate all interfaces and generate proxy code required for enforcing ownership discipline. The IDL also validates that cross-domain calls only consist of exchangeable types. When a domain crashes, i.e., a thread incurs a panic while in that domain, the domain enters the unwind routine and all subsequent calls to that domain will result in an Error but will not violate memory safety or panic. The unwind routine will restore the state of the domain to the call site eliminating the need for a continuation stack.

\chapter{Isolation Mechanisms}

\label{Chapter3}

\lhead{Chapter 3. \emph{Isolation Mechanisms}} 

\tirth{ Some intro here }

\section{Memory Protection Keys}

Recently, Intel introduced Memory Protection Keys for isolation support in the hardware. The 4 unused bits in the page table are used to assign a protection key to a page.  The PKRU register contains a 32-bit key representing the read/write permissions for 16 possible keys. The PKRU register can be accessed using user-mode \lstinline{WRPKRU} (for writing) and \lstinline{RDPKRU} instructions. For process sandboxing, the trusted computing base (TCB) can assign permissions to memory by writing the protection key to the PKRU register before passing it to the untrusted domain. To ensure an untrusted domain cannot use these instructions, we can use binary rewriting to remove all occurences of \lstinline{WRPKRU} and \lstinline{RDPKRU}.

Process isolation with MPK incurs a very low overhead. Crossing domains takes domains takes only 20-26 cycles \cite{ipc-62, ipc-33} and passing a buffer to an untrusted domain is simply manipulating bits in a global table that holds PKRU values for every domain. However, the number of domains that can co-exist is limited by the number of possible PKeys supported by the hardware.

\section{Native Client}
Introduced in 2009 by Google (now deprecated in favour of WebAssembly), Native Client (NaCl) added support for sandboxing untrusted native x86 code in browsers. NaCl limits the address space of a domain to a 4GiB segment. The two main invariants enforced by NaCl are:  no loads or stores can access data outside their 4GiB segment, and all jumps need to land to a valid instruction boundary inside the domain. The \lstinline{R15} register is reserved as \lstinline{RZP} which always points to the start of the domain. NaCl introduces pseudo-instructions that are finally expanded into x86 instructions that maintain the above invariants. 
All addresses are modified this way: the first 32 bits of the register are masked and then the register is added with R15. This way, every address lies in its respective segment. In addition to this, for jump instructions, the last 5 bits of the address register are masked so that the jump destination is always 32 bytes aligned. This is done to make sure that the masking instructions are not bypassed by jumping to an invalid target. 

The NaCl runtime incurs an overhead of only 10-15\%. Since, Google NaCl only supports a single untrusted domain, we adapt Google NaCl to support multiple domains. Data can be passed between domains by copying the buffer between segments. 

To evaluate how NaCl fares in a multi-domain setup, we implement a version that supports multiple domains. To compare the performance on ARM CPUs, we implement a similar setup in our evaluations.

\section{LXFI}
LXFI is a capability-based Software Fault Isolation mechanism that allows multiple untrusted domains to co-exist. A capability can be of three kinds: \lstinline{READ}, \lstinline{WRITE} and \lstinline{CALL} denoting read, write and execute permissions respectively. Capabilities are stored in a hash-table which binds a memory region with the domain that owns it. LXFI uses compile-time rewriting to enforce a capability check before every load/store operation and every call or indirect jump to ensure that the domain has the capability to access the data or call the function. 

\section{ARM NaCl}

\section{ARM Pointer Authentication}

\section{ARM Memory Tagging Extensions}

\section{Evaluating Isolation Mechanisms}
To evaluate the aforementioned isolation mechanisms, 

\chapter{The Network Stack} % Main chapter title

\label{Chapter4} % For referencing the chapter elsewhere, use \ref{Chapter1} 

\lhead{Chapter 4. The Network Stack} % This is for the header on each page - perhaps a shortened title

In this chapter, we take a look at how we implemented the UDP stack for Atmosphere. Since Atmosphere is not ready yet, we build a standalone network stack that runs in the userspace and evaluate its performance. We use a custom userspace network driver (Ixy) to process raw ethernet frames sent to and from the network stack. This driver is shared across applications just as it would be in the real Atmosphere.

\section{Driver for the Network Card}

Ixy is a network driver implemented in the userspace for Intel's Ixgbe family of network cards that shows how network cards work at the driver level. It is implemented in a fashion similar to DPDK and Snabb. We modify Ixy to fit the design of our network stack. Two main limitations of Ixy (as per our design) are: (1) it handles allocations in the driver and (2) received packets cannot be reused as transmit packets. (2) is important as for a lot of applications (e.g. MICA), we want to be able to flip headers in a received packet and send it back to the client. 

The major modifications we make to Ixy are:
\begin{itemize}
\item{Change the way packets are allocated. We get rid of the huge-pages backed Mempool and allow packets to be discretely allocated. DMA mapping is done for every packet. While this is wasteful in terms of memory (each packet takes a page), there is no discernible change in performance in processing packets. This also enables the NIC to interleave packets from different applications in the same queue without crossing any isolation boundaries. }
\item{Receive queues do not allocate packets themselves. Instead, the application supplies a batch of allocated buffers which is used to read packets from the NIC. This relieves the NIC from having to do any allocations. }

On testing with an Ixgbe NIC with a line rate of 10GB/s, we see no drop in performance after our 

\section{Network Manager}

The network manager is a shared component that manages and allocates ports to the applications. The network manager keeps track of open ports and associates network stacks with a port. It also holds a queue of empty buffers for each allocated stack. When a stack is not actively receiving packets from the NIC, it can buffer packets in a separate queue to relieve the NIC from buffering packets. This is done with the help of packet flipping. 

\section{Packet Processing}

\subsection*{Batching}

\subsection{Packet Flipping}
When a stack receives a batch of packets by calling \lstinline{rx_batch}, it can go through the batch and read the protocol and destination port and match them against its own protocol and port. If the tuple doesn't match, it can pass the buffer to the packet manager which will ``flip" the packet buffer with a pre-allocated empty buffer from an appropriate queue. This way, stacks can "help" other stacks receive packets from the same NIC.

\chapter{Evaluation} 

\label{Chapter5}

\lhead{Chapter 5. Evaluation} 
To evaluate the performance of our network stack, we run experiments on the Cloudlab network testbed. We utilize two c220g2 servers  configured with two Intel E5-2660 v3 10-core Haswell CPUs, 160 GB RAM, and a dual-port Intel X520 10Gb NIC. We disable hyperthreading and set the CPUs to a constant frequency of 2.6GHz to reduce the variance in benchmarking. 

\section{L2 Forward}
We use DPDK's L2Forward example and benchmark it against a similar implementation in Rust. We vary the packet size from 100-1500 in steps of 100 with a batch size of 32 for both the implementations and sent packets at a rate of 10Gb/s (the line rate of the NIC). Since the CPU can easy saturate a 10Gb/s line while running at 2.6GHz, we also compare the performance at a frequency of 1GHz. 

\begin{figure}[!htbp]
	\includegraphics[width=1.0\columnwidth]{figures/l2fwd26.pdf}
\caption{L2Forward throughput measured at 2.6GHz}
	\label{fig:l2fwd26}
\end{figure}

\tirth{Add 1GHz benchmark for Ixy}

\section{Key-Value Store}
To study the overheads of our network stack on a realistic workload, we implement an in-memory key-value store backed by a hash-table based on the Fowler-Noll-Vo algorithm. Processing a packet consists of parsing the request, performing an insert/fetch operation in the hash table, filling the response buffer queuing the packet in the send batch. Transmitting a packet takes fewer cycles than receiving because calls to \lstinline{rx_batch} wait till the NIC is done writing packets into the DMA buffers while \lstinline{tx_batch} simply holds the DMA buffers in a private queue and the NIC can write to them asynchronously. The next time \lstinline{tx_batch} is called, the queue is cleaned and the buffers are returned as ``free buffers".

\begin{table}[!htbp]
    \begin{small}
    \begin{center}
  \begin{tabular}{| c | c | c | c | c | c |}
  \hline
  capacity (power of 2)   & operations (power of 2)   & rx\textunderscore tsc\textbackslash pkt  &  process\textunderscore tsc\textbackslash pkt    & tx\textunderscore tsc\textbackslash pkt    \\
  \hline
  \hline
  16 & 14 & 763 & 911 & 354 \\
  18 & 16 & 807 & 813 & 363 \\
  20 & 18 & 811 & 809 & 363 \\
  22 & 20 & 813 & 805 & 364 \\
  24 & 22 & 814 & 804 & 364 \\
  26 & 24 & 814 & 804 & 364 \\
  28 & 26 & 813 & 805 & 364 \\
  \hline
\end{tabular}
\end{center}
\end{small}
\caption{Microarchitectural comparison of SFI and C on FFmpeg-x86 benchmark.}
\label{table:micro-ffmpeg-x86}
\end{table}
%\input{Chapters/Chapter6}
%\input{Chapters/Chapter7}

%-------------------------------------------------------------------------------
%	THESIS CONTENT - APPENDICES
%-------------------------------------------------------------------------------

\addtocontents{toc}{\vspace{2em}} % Add a gap in the Contents, for aesthetics

\appendix % Cue to tell LaTeX that the following 'chapters' are Appendices

% Include the appendices of the thesis as separate files from the Appendices
% folder
% Uncomment the lines as you write the Appendices

% \input{Appendices/AppendixA}
%\input{Appendices/AppendixB}
%\input{Appendices/AppendixC}

\addtocontents{toc}{\vspace{2em}} % Add a gap in the Contents, for aesthetics

\backmatter

%-------------------------------------------------------------------------------
%	BIBLIOGRAPHY
%-------------------------------------------------------------------------------

\label{Bibliography}

\lhead{\emph{Bibliography}} % Change the page header to say "Bibliography"

\printbibliography

\end{document}
