% Chapter 1

\chapter{Introduction} % Main chapter title

\label{Chapter1} % For referencing the chapter elsewhere, use \ref{Chapter1} 

\lhead{Chapter 1. Introduction} % This is for the header on each page - perhaps a shortened title

Operating system reliability is a topic that has been studied for decades but remains a major concern till date. Even thought its been 30 years since Linux kernel was developed, critical vulnerabilities are being found till date. On top of that, the monolithic design (ie everything in the kernel running in a single address space) makes it such that the attacker is only one exploit away from taking control of the entire operating system. This is concerning because device drivers, often developed by third parties are a major source of vulnerabilities. Redundancy in hardware and software protection mechanisms can protect against transient faults but persistent faults because of undetected logic flaws still remain a major concern. In recent times, with the advent of cloud services, this becomes even more significant because clouds often have to run untrusted code that needs to be isolated from other parts of the operating system.

Logic errors and faults can propogate from within a component and propogate to other parts of the system. Microkernelization aims to solve this problem by isolating various parts of an operating system and minimizing the effect of failures in a single subsystem. In a microkernel, only the bare minimum that is required to boot an operating system is included in the kernel. All other components such as the filesystem, the device drivers, and the network stack are run in separate isolated processes that interact with each other using some form of IPC. Inter component fault propogation can thus be reduced by introducing safe IPC mechanisms.

Atmosphere is a microkernel based operating system. In this thesis, I discuss the design of Atmosphere mainly focusing on the the network stack and how its design can be used to model other services of the operating system. The network stack was built two main principles in mind: maximizing fault isolation and minimizing its cost on performance. For the first principle, we take microkernelization to its extremes and build a per connection (or per socket for UDP) network stack. This means, every new connection has its own TCP/IP stack that can process and dispatch packets. This ensures that if one connection is corrupted, the rest of the connections on the system can keep operating. Although this is not completely possible since there is some inherent shared state on any host machine on a network. In later sections we discuss how we minimize this shared state. For the second principle, we implement a zero copy model and explain how using Rust's guarantees help us implement this with confidence. On failure, a component can be safely restarted and meanwhile, all IPC calls to it can return an error. 

The rest of this Thesis is organised as follows: 

\begin{itemize}
    \item{In \nameref{Chapter2} we take a look at the contemporary approaches to microkernelization. }
    \item{\nameref{Chapter3} discusses the design principles of Atmosphere.}
    \item{\nameref{Chapter4} talks about the API design and the internals of the network stack we built.  }
    \item{\nameref{Chapter5} evaluates the performance of the network stack as compared to the current standards.}
\end{itemize}
