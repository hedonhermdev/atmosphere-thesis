\chapter{The Network Stack} % Main chapter title

\label{Chapter4} % For referencing the chapter elsewhere, use \ref{Chapter1} 

\lhead{Chapter 4. The Network Stack} % This is for the header on each page - perhaps a shortened title

In this chapter, we take a look at how we implemented the UDP stack for Atmosphere. Since Atmosphere is not ready yet, we build a standalone network stack that runs in the userspace and evaluate its performance. We use a custom userspace network driver (Ixy) to process raw ethernet frames sent to and from the network stack. This driver is shared across applications just as it would be in the real Atmosphere.

\section{Driver for the Network Card}

Ixy is a network driver implemented in the userspace for Intel's Ixgbe family of network cards that shows how network cards work at the driver level. It is implemented in a fashion similar to DPDK and Snabb. We modify Ixy to fit the design of our network stack. Two main limitations of Ixy (as per our design) are: (1) it handles allocations in the driver and (2) received packets cannot be reused as transmit packets. (2) is important as for a lot of applications (e.g. MICA), we want to be able to flip headers in a received packet and send it back to the client. 

The major modifications we make to Ixy are:
\begin{itemize}
\item{Change the way packets are allocated. We get rid of the huge-pages backed Mempool and allow packets to be discretely allocated. DMA mapping is done for every packet. While this is wasteful in terms of memory (each packet takes a page), there is no discernible change in performance in processing packets. This also enables the NIC to interleave packets from different applications in the same queue without crossing any isolation boundaries. }
\item{Receive queues do not allocate packets themselves. Instead, the application supplies a batch of allocated buffers which is used to read packets from the NIC. This relieves the NIC from having to do any allocations. }

On testing with an Ixgbe NIC with a line rate of 10GB/s, we see no drop in performance after our 

\section{Network Manager}

The network manager is a shared component that manages and allocates ports to the applications. The network manager keeps track of open ports and associates network stacks with a port. It also holds a queue of empty buffers for each allocated stack. When a stack is not actively receiving packets from the NIC, it can buffer packets in a separate queue to relieve the NIC from buffering packets. This is done with the help of packet flipping. 

\section{Packet Processing}

\subsection*{Batching}

\subsection{Packet Flipping}
When a stack receives a batch of packets by calling \lstinline{rx_batch}, it can go through the batch and read the protocol and destination port and match them against its own protocol and port. If the tuple doesn't match, it can pass the buffer to the packet manager which will ``flip" the packet buffer with a pre-allocated empty buffer from an appropriate queue. This way, stacks can "help" other stacks receive packets from the same NIC.
