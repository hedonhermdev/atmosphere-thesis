\chapter{Evaluation} 

\label{Chapter5}

\lhead{Chapter 5. Evaluation} 
To evaluate the performance of our network stack, we run experiments on the Cloudlab network testbed. We utilize two c220g2 servers  configured with two Intel E5-2660 v3 10-core Haswell CPUs, 160 GB RAM, and a dual-port Intel X520 10Gb NIC. We disable hyperthreading and set the CPUs to a constant frequency of 2.6GHz to reduce the variance in benchmarking. 

\section{L2 Forward}
We use DPDK's L2Forward example and benchmark it against a similar implementation in Rust. We vary the packet size from 100-1500 in steps of 100 with a batch size of 32 for both the implementations and sent packets at a rate of 10Gb/s (the line rate of the NIC). Since the CPU can easy saturate a 10Gb/s line while running at 2.6GHz, we also compare the performance at a frequency of 1GHz. 

\begin{figure}[!htbp]
	\includegraphics[width=1.0\columnwidth]{figures/l2fwd26.pdf}
\caption{L2Forward throughput measured at 2.6GHz}
	\label{fig:l2fwd26}
\end{figure}

\tirth{Add 1GHz benchmark for Ixy}

\section{Key-Value Store}
To study the overheads of our network stack on a realistic workload, we implement an in-memory key-value store backed by a hash-table based on the Fowler-Noll-Vo algorithm. Processing a packet consists of parsing the request, performing an insert/fetch operation in the hash table, filling the response buffer queuing the packet in the send batch. Transmitting a packet takes fewer cycles than receiving because calls to \lstinline{rx_batch} wait till the NIC is done writing packets into the DMA buffers while \lstinline{tx_batch} simply holds the DMA buffers in a private queue and the NIC can write to them asynchronously. The next time \lstinline{tx_batch} is called, the queue is cleaned and the buffers are returned as ``free buffers".

\begin{table}[!htbp]
    \begin{small}
    \begin{center}
  \begin{tabular}{| c | c | c | c | c | c |}
  \hline
  capacity (power of 2)   & operations (power of 2)   & rx\textunderscore tsc\textbackslash pkt  &  process\textunderscore tsc\textbackslash pkt    & tx\textunderscore tsc\textbackslash pkt    \\
  \hline
  \hline
  16 & 14 & 763 & 911 & 354 \\
  18 & 16 & 807 & 813 & 363 \\
  20 & 18 & 811 & 809 & 363 \\
  22 & 20 & 813 & 805 & 364 \\
  24 & 22 & 814 & 804 & 364 \\
  26 & 24 & 814 & 804 & 364 \\
  28 & 26 & 813 & 805 & 364 \\
  \hline
\end{tabular}
\end{center}
\end{small}
\caption{Microarchitectural comparison of SFI and C on FFmpeg-x86 benchmark.}
\label{table:micro-ffmpeg-x86}
\end{table}